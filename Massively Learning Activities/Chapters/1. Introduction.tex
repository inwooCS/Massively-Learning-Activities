\section{Introduction} \label{section: introduction}
\textcolor{red}{\lipsum[1]}
\subsection{TASI/PHIDC}

\textbf{ABOUT US}
\\
The Telecommunications and Social Informatics Research Program / Pacific Health Informatics and Data Center (TASI/PHIDC), formerly TASI/PEACESAT, is part of the Social Science Research Institute (SSRI) of the College of Social Sciences (CSS) at the University of Hawai‘i at Manoa. TASI/PHIDC programs incorporate an interdisciplinary approach to education and research, and work with partners from across the University of Hawai’i system, State of Hawai’i and other government and academic institutions from the Asia and Pacific Islands region. Program and research focus areas include policy, planning, information and communications technologies and systems, health information technology, health informatics in Hawai‘i and the Pacific Islands region.

\textbf{MISSION}
\\
The TASI/PHIDC Research Program missions are to: (1) Provide technical assistance in policy, program planning and evaluation; (2) Facilitate public and private sector collaboration to improve community resiliency, sustainability, and health system performance; and (3) Build capacity in information technology, health data management, analytics, and data sciences.

\textbf{FACULTY RESEARCH}
\\
TASI/PHIDC conducts interdisciplinary and applied research and provides policy, program, technical assistance, education, and training in Hawai‘i and the Pacific Islands Region related to:

\begin{itemize}
    \item Accessible and affordable Information and Communication Technology (ICT)
    \item Health Information Technology (HIT)
    \item Electronic Health Record (EHR)
    \item Healthcare and claims data management, analytics, and programs
    \item Telehealth
    \item Meteorological and disaster communications
\end{itemize}

\subsection{TASI \& CNMI (Contract Explained)}
TASI/PHIDC is a Technical Assistance and Research Partner or “TARP” who has an Intergovernmental Cooperative Agreement (ICA) with the Commonwealth of the Northern Mariana Islands (CNMI) State Medicaid Agency (SMA) to design an infrastructure that would allow advanced data analytics and parallel processing of Protected Health Information. After careful consideration, TASI/PHIDC has opted for SAS technologies in a hyper-converged infrastructure.

\begin{itemize}
    \item \textcolor{red}{Modernize data archive and storage (paper to electronic) of PHI data.}
    \item \textcolor{red}{Want to perform data analytics and machine learning.}
    \item \textcolor{red}{Used RCUH funds to purchase SAS license.}
    \item \textcolor{red}{Therefore, SAS needs to be accessible to multi-tenants and UH themselves.}
\end{itemize}

\subsection{TASI \& SAS (Contract Summarized)}

\begin{enumerate} 
    \item Pre-Deployment and Project Management (ETC 14 Hours)
    \begin{itemize}
        \item Before deploying SAS technologies, TASI and SAS will engage in pre-deployment and project management tasks. 
        \item These tasks will involve ongoing project management to ensure that the project plan is followed, and appropriate resources are assigned. The project plan will include details of billable work hours logs that will be sent by SAS and verified by UHTASI. In addition, SAS will send Pre-Install Requirements Documents to UHTASI for completion, and UHTASI will review the completion of these documents to ensure environmental readiness for installation. These tasks will require an estimated 14 hours of work.
    \end{itemize}
    \item Deployment (ETC 70 Hours)
    \begin{itemize}
        \item During the deployment phase, TASI will receive the installation of several SAS products:
        \begin{itemize}
            \item SAS Advanced Analytics for Education (on Viya 3.5)
            \item SAS Data Preparation
            \item SAS Data Management Advanced
            \item SAS Education Analytical Suite
            \item SAS Text Analytics for Education
        \end{itemize}
        
        \item Configuration will also be performed, which includes establishing a database connection and testing it. A validation of the new environment will be conducted to ensure that all components are working as intended before the handoff. Data libraries will be created, and SAS user access controls will be established. TASI will also verify that each of the server components is active and is handling requests. Finally, SAS will provide TASI with installation documentation.
    \end{itemize}
\end{enumerate}
