\section{Massively Learning Activities II - Migration Deployment} \label{section: MLA2}
TASI has been contracted by CNMI to create an infrastructure that allows for data analytics on Protected Health Information (PHI). This infrastructure will initially be hosted on-premises, with plans to move towards a hybrid solution in the future. To achieve this, we will be providing a Platform as a Service (PaaS) solution, by hosting SAS Viya services on our own hardware and allowing tenants to access and utilize the platform for their own analytics applications.

The tenants, including APCD, CMNI, CMA, Criminal Justice, and several Education environments, will provide the necessary data, which will be submitted to an ETL data pipeline for processing before being sent to SAS on-prem servers. Once the data has been processed, tenants may perform data analytics using advanced algorithms in SAS programming language.

To ensure secure operations, we will configure the security relationships between the software, hardware, and tenants using LDAP, security groups, encryption  and other related tools. Our goal is to architect a high-performance infrastructure that allows for advanced data analytics while maintaining the confidentiality and security of PHI.

Due to SAS being a time sensitive project, the initial deployment will have SAS suites and VMs installed on existing hardware, with plans to migrate the infrastructure to newly acquired hardware in the future.

The final and completed deployment of SAS Viya 3.5 will expect a total of 8 tenants:

\begin{enumerate}
    \item Commonwealth of the Northern Mariana Islands (CNMI)
    \item All-Payer Claims Database (APCD)
    \item Centers for Medicare \& Medicaid Services (CMA)
    \item Med-Quest
    \item University Education 1
    \item University Education 2
    \item University Education 3
    \item University Education 4
\end{enumerate}

\subsection{Planning II}

The System Development Lifecycle (SDLC) is a project management model that defines different stages that are necessary to bring a project from conception to deployment and later maintenance. The SDLC model consists of several phases, which typically include requirements gathering, design, development, testing, deployment, and maintenance. The specific activities within each phase may vary depending on the project and the organization, but the basic principles are the same. The SDLC model is a flexible framework that can be adapted to suit the needs of different projects and organizations. It provides a systematic approach to software development that helps ensure that software is built efficiently, effectively, and with minimal risk.

Massively Learning Activities will follow a similar variation to the SDLC project management model where each SDLC stage will correspond to a subsection in this chapter. 

\subsection{Required of Analysis II}
\begin{itemize}
    \item Requirement of Analysis
    \item Deployment Design
\end{itemize}

\subsection{Design II}
We will be using VMotion to migrate virtual machines. 

\subsection{Implementation II}

\subsection{Testing \& Integration II}

\subsection{Operations \& Maintenance II}

