\section*{Abstract}

This paper provides an overview of UHTASI's comprehensive project with SAS technologies, which encompasses various aspects and involves partnerships with multiple agencies in the Pacific region. The project's primary objective is to establish an infrastructure for deploying SAS technologies to perform data analytics on sensitive PHI data, including healthcare claims and criminal justice records. Recognizing the urgency of the project, UHTASI has made concerted efforts to expedite the deployment of SAS technologies, focusing on their multi-tenancy capabilities and utilizing existing on-premises hardware.

Following a successful initial deployment, UHTASI plans to acquire additional hardware and migrate the existing SAS infrastructure to a new hyper-converged infrastructure (HCI) developed by UHTASI. Once the migration to HCI is complete, UHTASI aims to replicate the HCI environment and incorporate four additional tenants, enabling faculty and students at the University to access SAS technologies for research and educational purposes. Notably, the HCI instance for the University will be segregated from the original HCI, ensuring separate and secure environments.

This paper highlights the progressive steps taken by UHTASI in deploying SAS technologies, emphasizing the efficient utilization of resources and collaborative partnerships. The successful implementation of this project will contribute to improved data analytics capabilities and facilitate academic engagement with SAS technologies in the University setting.