\section{Massively Learning Activities II - Migration Deployment} \label{section: MLA2}
Although project development plans can differ, they commonly follow a similar framework known as the SDLC (System Development Life Cycle). The System Development Life-cycle (SDLC) is a project management model that defines different stages that are necessary to bring a project from conception to deployment and later maintenance. 

The SDLC model consists of several phases: planning, research, design, implementation, testing \& integration, and maintenance. It provides a systematic approach to system development that helps ensure that system is built efficiently with minimal risk.

We explore the logistics of configuring a multi-tenant hyper converged infrastructure by documenting the entire project management process, inspired by the System Development Life-Cycle framework.  Massively Learning Activities will follow a similar variation of the SDLC project management model where each SDLC stage will correspond to a subsection in this chapter. 

The final and completed deployment of SAS Viya 3.5 will expect a total of 8 tenants:

\begin{enumerate}
    \item Commonwealth of the Northern Mariana Islands (CNMI)
    \item All-Payer Claims Database (APCD)
    \item Centers for Medicare \& Medicaid Services (CMA)
    \item Med-Quest
    \item University Education 1
    \item University Education 2
    \item University Education 3
    \item University Education 4
\end{enumerate}

\subsection{Planning II}

The System Development Lifecycle (SDLC) is a project management model that defines different stages that are necessary to bring a project from conception to deployment and later maintenance. The SDLC model consists of several phases, which typically include requirements gathering, design, development, testing, deployment, and maintenance. The specific activities within each phase may vary depending on the project and the organization, but the basic principles are the same. The SDLC model is a flexible framework that can be adapted to suit the needs of different projects and organizations. It provides a systematic approach to software development that helps ensure that software is built efficiently, effectively, and with minimal risk.

Massively Learning Activities will follow a similar variation to the SDLC project management model where each SDLC stage will correspond to a subsection in this chapter. 

\subsection{Required of Analysis II}
\begin{itemize}
    \item Requirement of Analysis
    \item Deployment Design
\end{itemize}

\subsection{Design II}
We will be using VMotion to migrate virtual machines. 

\subsection{Implementation II}

\subsection{Testing \& Integration II}

\subsection{Operations \& Maintenance II}

