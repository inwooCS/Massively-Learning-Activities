\section{Introduction} \label{section: introduction}
\textcolor{red}{\lipsum[1]}
\subsection{TASI/PHIDC}

\textbf{ABOUT US}
\\
The Telecommunications and Social Informatics Research Program / Pacific Health Informatics and Data Center (TASI/PHIDC), formerly TASI/PEACESAT, is part of the Social Science Research Institute (SSRI) of the College of Social Sciences (CSS) at the University of Hawai‘i at Manoa. TASI/PHIDC programs incorporate an interdisciplinary approach to education and research, and work with partners from across the University of Hawai’i system, State of Hawai’i and other government and academic institutions from the Asia and Pacific Islands region. Program and research focus areas include policy, planning, information and communications technologies and systems, health information technology, health informatics in Hawai‘i and the Pacific Islands region.

\textbf{MISSION}
\\
The TASI/PHIDC Research Program missions are to: (1) Provide technical assistance in policy, program planning and evaluation; (2) Facilitate public and private sector collaboration to improve community resiliency, sustainability, and health system performance; and (3) Build capacity in information technology, health data management, analytics, and data sciences.

\textbf{FACULTY RESEARCH}
\\
TASI/PHIDC conducts interdisciplinary and applied research and provides policy, program, technical assistance, education, and training in Hawai‘i and the Pacific Islands Region related to:

\begin{itemize}
    \item Accessible and affordable Information and Communication Technology (ICT)
    \item Health Information Technology (HIT)
    \item Electronic Health Record (EHR)
    \item Healthcare and claims data management, analytics, and programs
    \item Telehealth
    \item Meteorological and disaster communications
\end{itemize}

\subsection{TASI \& CNMI (Contract Explained)}
TASI/PHIDC is a Technical Assistance and Research Partner or “TARP” who has an Intergovernmental Cooperative Agreement (ICA) with the Commonwealth of the Northern Mariana Islands (CNMI) State Medicaid Agency (SMA) to design an infrastructure that would allow advanced data analytics and parallel processing of Protected Health Information. After careful consideration, TASI/PHIDC has opted for SAS technologies in a hyper-converged infrastructure.

\begin{itemize}
    \item \textcolor{red}{Modernize data archive and storage (paper to electronic) of PHI data.}
    \item \textcolor{red}{Want to perform data analytics and machine learning.}
    \item \textcolor{red}{Used RCUH funds to purchase SAS license.}
    \item \textcolor{red}{Therefore, SAS needs to be accessible to multi-tenants and UH themselves.}
\end{itemize}

\subsection{TASI \& SAS (Contract Explained)}

\begin{enumerate} 
    \item Pre-Deployment and Project Management (ETC 14 Hours)
    \begin{itemize}
        \item Ongoing Project Management Tasks
        \item Prepare project plan and assign appropriate resources. Project plan should include: 
        \begin{itemize}
            \item SAS will send billable work hours log
            \item UHTASI will verify those working hours on project plan
        \end{itemize}
        \item Send Pre-Install Requirements Documents to UHTASI for completion
        \item Ensure environmental readiness for install by reviewing the completion of the Pre-Install Requirements Document
    \end{itemize}
    \item Deployment (ETC 70 Hours)
    \begin{itemize}
        \item Installation of the following products in a single environment:
        \begin{itemize}
            \item SAS Advanced Analytics for Education (on Viya 3.5)
            \item SAS Data Preparation
            \item SAS Data Management Advanced
            \item SAS Education Analytical Suite
            \item SAS Text Analytics for Education
        \end{itemize}
        \item Configuration
        \begin{itemize}
            \item Establish database connection and test
        \end{itemize}
        \item Preform validation of new environment to ensure all components are working as intended before handoff
        \begin{itemize}
            \item Creation of data libraries
            \item SAS User Access Controls
        \end{itemize}
        \item Verify each of the server components is active and is handling requests
        \item Provide installation documentation
    \end{itemize}
\end{enumerate}
